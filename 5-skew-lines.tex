\question % 2015 Exam, Q10.
Let $A(-4, 0, 2)$ and $B(6, 10, 3)$ be points and let $\vec{u} = \colvec{3}{2}{2}{-1}$ and $\vec{v} = \colvec{3}{-2}{1}{-2}$ be vectors.

\begin{parts}

\part[1]
Find $\vv{AB}$.

\begin{EnvFullwidth}
\begin{solutionorgrid}[1in]
We have
\[
    \vv{AB} = \colvec{3}{6 - (-4)}{10 - 0}{3 - 2} = \colvec{3}{10}{10}{1}.
\]
\end{solutionorgrid}
\end{EnvFullwidth}

\part[2]
Find $\vec{u} \times \vec{v}$.

\begin{EnvFullwidth}
\begin{solutionorgrid}[3in]
By technology
\begin{align*}
    \vec{u} \times \vec{v} &= \colvec{3}{2}{2}{-1} \times \colvec{3}{-2}{1}{-2} \\
    &= \colvec{3}{-3}{6}{6}.
\end{align*}
\end{solutionorgrid}
\end{EnvFullwidth}

\part[2]
Show that $\displaystyle{\frac{\abs{\vv{AB} \cdot \vec{u} \times \vec{v}}}{\abs{\vec{u} \times \vec{v}}} = 4}$.

\begin{EnvFullwidth}
\begin{solutionorgrid}[2in]
By part (b)
\[
    \abs{\vec{u} \times \vec{v}} = \sqrt{9 + 36 + 36} = 9,
\]
and by (a) and (b)
\[
    \abs{\vv{AB} \cdot \vec{u} \times \vec{v}} = \abs{\colvec{3}{10}{10}{1} \cdot \colvec{3}{-3}{6}{6}} = \abs{-30 + 60 + 6} = 36.
\]
Thus,
\begin{align*}
    \frac{\abs{\vv{AB} \cdot \vec{u} \times \vec{v}}}{\abs{\vec{u} \times \vec{v}}} &= \frac{36}{9} \\
    &= 4.
\end{align*}
\end{solutionorgrid}
\end{EnvFullwidth}

\ifprintanswers
\else
\newpage
\fi
\uplevel{Line $\ell_1$ contains point $A$ and has direction vector $\vec{u}$. Line $\ell_1$ is in the plane
\[
    P_1 : -x + 2y + 2z = 8.
\]
Let $t$ be a real number and let line $\ell_2$ have the parametric equations
\[
    \begin{cases}
        x = 6 - 2t \\
        y = 10 + t \\
        z = 3 - 2t.
    \end{cases}
\]
}

\part[1]
Show that $\ell_2$ is parallel to $P_1$.

\begin{EnvFullwidth}
\begin{solutionorgrid}[1.5in]
A direction vector for $\ell_2$ and a normal vector for $P_1$ are
\[
    \vec{v} = \colvec{3}{-2}{1}{-2}, \qquad \vec{n_{1}} = \colvec{3}{-1}{2}{2}.
\]
But since $\vec{v} \cdot \vec{n_1} = 0$, we have
\[
    \vec{v} \perp \vec{n_1} \implies \ell_2 \parallel P_1.
\]
\end{solutionorgrid}
\end{EnvFullwidth}

\part[2]
Show that $\ell_2$ is in the plane
\[
    P_2 : -x + 2y + 2z = 20.
\]

\begin{EnvFullwidth}
\begin{solutionorgrid}[2in]
On substitution of $\ell_2$ into $P_2$ we get
\begin{align*}
    -(6 - 2t) + 2(10 + t) + 2(3 - 2t) &= -6 + 20 + 6 + 2t + 2t - 4t \\
    &= 20.
\end{align*}
So, $\ell_2$ satisfies the equation of $P_2$.
\end{solutionorgrid}
\end{EnvFullwidth}

\part[2]
Find the distance $d$ between the parallel planes $P_1$ and $P_2$.

\begin{EnvFullwidth}
\begin{solutionorgrid}[2in]
Since $P_2 : -x + 2y + 2z - 20 = 0$ but $A(-4, 0, 2)$ is in $P_1$, by the distance from a point to a plane formula
\begin{align*}
    d &= \frac{\abs{-1 \times -4 + 2 \times 0 + 2 \times 2 - 20}}{\sqrt{(-1)^2 + 2^2 + 2^2}} \\
    &= \frac{12}{3} \\
    &= 4 \, \textrm{units}.
\end{align*}
\end{solutionorgrid}
\end{EnvFullwidth}

\part[2]
Explain why $\ell_1$ and $\ell_2$ are skew lines.

\begin{EnvFullwidth}
\begin{solutionorgrid}[1in]
The shortest distance between $\ell_1$ and $\ell_2$ is $4 > 0$, so they don't intersect. But since $\ell_1 \nparallel \ell_2$, it follows that $\ell_1$ and $\ell_2$ are skew.
\end{solutionorgrid}
\end{EnvFullwidth}

\part[1]
State the distance between $\ell_1$ and $\ell_2$. \emph{Hint}: consider Theorem~\ref{thm:shortest-distance-between-skew-lines}.

\begin{EnvFullwidth}
\begin{solutionorgrid}[.75in]
The distance between $\ell_1$ and $\ell_2$ is $4$~units.
\end{solutionorgrid}
\end{EnvFullwidth}

\uplevel{Let $s$ be a real number and let line $\ell_3$ have the parametric equations
\[
    \begin{cases}
        x = 2 - 2s \\
        y = -1 + s \\
        z = -2s.
    \end{cases}
\]
}

\part[1]
Find the equation of plane $P_3$ which contains $\ell_3$ and is parallel to $P_1$.

\begin{EnvFullwidth}
\begin{solutionorgrid}[1in]
We have
\[
    P_3 : -x + 2y + 2z = k.
\]
On substitution
\[
    -(2 - 2s) + 2(-1 + s) + 2(-2s) = -4.
\]
Thus,
\[
    P_3 : -x + 2y + 2z = -4.
\]
\end{solutionorgrid}
\end{EnvFullwidth}

% Note: there's a picture that I need to put in here one of these days, but it's too tricky for now.

\part[2]
Explain whether $\ell_2$ and $\ell_3$ are on the same side of $P_1$, or on opposite sides.

\begin{EnvFullwidth}
\begin{solutionorgrid}[1in]
Lines $\ell_2$ and $\ell_3$ are on opposite sides of $P_1$ as $-2 < 4 < 10$ so $P_1$ is between $P_2$ and $P_3$.
\end{solutionorgrid}
\end{EnvFullwidth}

\end{parts}

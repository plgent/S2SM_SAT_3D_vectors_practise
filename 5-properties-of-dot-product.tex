\question % 2013 Exam, Q6.
Let $\vec{a} = \colvec{3}{1}{2}{-4}$ and $\vec{b} = \colvec{3}{5}{-3}{-2}$ be vectors.

\begin{parts}

\part

\begin{subparts}

\subpart[1]
Find $\abs{\vec{a}}$.

\begin{EnvFullwidth}
\begin{solutionorgrid}[1in]
We have
\[
    \abs{\vec{a}} = \sqrt{1^2 + 2^2 + (-4)^2} = \sqrt{21}.
\]
\end{solutionorgrid}
\end{EnvFullwidth}

\subpart[1]
Find $\vec{a} \cdot \vec{b}$.

\begin{EnvFullwidth}
\begin{solutionorgrid}[1in]
We have
\[
    \vec{a} \cdot \vec{b} = 1 \times 5 + 2 \times (-3) + (-4) \times (-2) = 7.
\]
\end{solutionorgrid}
\end{EnvFullwidth}

\subpart[2]
Verify that $\abs{\vec{a} + \vec{b}}^2 = \abs{\vec{a}}^2 + \abs{\vec{b}}^2 + 2\vec{a} \cdot \vec{b}$.

\begin{EnvFullwidth}
\begin{solutionorgrid}[2.5in]
The LHS is
\[
    (\sqrt{6^2 + (-1)^2 + (-6)^2})^2 = 73,
\]
and, by parts (a)(i) and (ii), the RHS is
\[
    (\sqrt{21})^2 + (\sqrt{38})^2 + 2 \times 7 = 73.
\]
\end{solutionorgrid}
\end{EnvFullwidth}

\end{subparts}

\part[2]
Use Proposition~\ref{prop:dot-product-of-vector-sum-with-itself} to show that $\abs{\vec{u} + \vec{v}}^2 = \abs{\vec{u}}^2 + \abs{\vec{v}}^2 + 2\vec{u} \cdot \vec{v}$ for \emph{all} vectors $\vec{u}$ and $\vec{v}$.

\begin{EnvFullwidth}
\begin{solutionorgrid}[2.5in]
\begin{proof}
Let $\vec{u}$ and $\vec{v}$ be vectors. By Proposition~\ref{prop:dot-product-of-vector-sum-with-itself}
\begin{align*}
    \abs{\vec{u} + \vec{v}}^2 &= (\vec{u} + \vec{v}) \cdot (\vec{u} + \vec{v}) \\
    &= \vec{u} \cdot \vec{u} + \vec{u} \cdot \vec{v} + \vec{v} \cdot \vec{u} + \vec{v} \cdot \vec{v} \\
    &= \abs{\vec{u}}^2 + \abs{\vec{v}}^2 + 2\vec{u} \cdot \vec{v}.
\end{align*}
\end{proof}
\end{solutionorgrid}
\end{EnvFullwidth}

\part
If $\abs{\vec{u}} = 10$, $\abs{\vec{v}} = 5$ and $\vec{u} \cdot \vec{v} = 25$, find:

\begin{subparts}

\subpart[1]
$\abs{\vec{u} + \vec{v}}$.

\begin{EnvFullwidth}
\begin{solutionorgrid}[1.5in]
By part (b)
\begin{align*}
    \abs{\vec{u} + \vec{v}} &= \sqrt{\abs{\vec{u}}^2 + \abs{\vec{v}}^2 + 2\vec{u} \cdot \vec{v}} && (\abs{\vec{u} + \vec{v}} \geq 0) \\
    &= \sqrt{10^2 + 5^2 + 2 \times 25} \\
    &= 5\sqrt{7}.
\end{align*}
\end{solutionorgrid}
\end{EnvFullwidth}

\subpart[1]
$\abs{\vec{u} - \vec{v}}$.

\begin{EnvFullwidth}
\begin{solutionorgrid}[1.5in]
Similarly,
\begin{align*}
    \abs{\vec{u} - \vec{v}} &= \sqrt{\abs{\vec{u}}^2 + \abs{\vec{v}}^2 - 2\vec{u} \cdot \vec{v}} && (\abs{\vec{u} - \vec{v}} \geq 0) \\
    &= \sqrt{10^2 + 5^2 - 2 \times 25} \\
    &= 5\sqrt{3}.
\end{align*}
\end{solutionorgrid}
\end{EnvFullwidth}

\end{subparts}

\end{parts}

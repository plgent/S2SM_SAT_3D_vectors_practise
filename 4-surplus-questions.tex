%%% Surplus questions

%%% 2004, Q13 is excellent.
%%% 2005, Q11 is too.
%%% Sample 1, Q3.
%%% Studies 2007, Q5.
%%% Studies 2015, Q9.

\question % 2017 Exam, Q12.
Consider the system of equations:
\[
	\systeme{
		x + 2y + 2z = 4,
		2x + y - 2z = 5,
		3x + 2y - 2z = 8
	}
\]

\begin{parts}

\part[1]
Write this system as an augmented matrix.

\begin{EnvFullwidth}
\begin{solutionorgrid}[1in]
Foo.
\end{solutionorgrid}
\end{EnvFullwidth}

\part[3]
Stating all row operations, show that the system has the solutions
\begin{align*}
	x &= 2 + 2t \\
	y &= 1 - 2t \\
	z &= t,
\end{align*}
where $t$ is a real number.

\begin{EnvFullwidth}
\begin{solutionorgrid}[4in]
Foo.
\end{solutionorgrid}
\end{EnvFullwidth}

% \uplevel{Let $P_1$, $P_2$ and $P_3$ be the three planes given by
% \[
% 	\sysdelim.. \systemeL{
% 		x + 2y + 2z = 4 @ P_1 :,
% 		2x + y - 2z = 5 @ P_2 :,
% 		3x + 2y - 2z = 8 @ P_3 :
% 	}
% \]
% }

\part[1]
Using part (b), show that $A(2, 1, 0)$ and $B(0, 3, -1)$ are common to all three planes.

\begin{EnvFullwidth}
\begin{solutionorgrid}[1.5in]
Foo.
\end{solutionorgrid}
\end{EnvFullwidth}

\part[3]
Show that $P_1$ and $P_2$ are perpendicular.

\begin{EnvFullwidth}
\begin{solutionorgrid}[2in]
Foo.
\end{solutionorgrid}
\end{EnvFullwidth}

\uplevel{The point $C(0, 6, 2)$ is on $P_3$ and $D(12, -4, 0)$ is on $P_1$. The line $\ell$ through $C$ and $D$ intersects $P_2$ at $E$, as shown in Figure~\ref{fig-three-planes}.}

\begin{figure}[h]
	\centering
% 	\includegraphics[scale=.5]{0-three-planes}
	\caption{Three planes.}
	\label{fig-three-planes}
\end{figure}

\part[2]
Find the equation of $\ell$ in parametric form.

\begin{EnvFullwidth}
\begin{solutionorgrid}[1in]
Foo.
\end{solutionorgrid}
\end{EnvFullwidth}

\part[2]
Find the coordinates of $E$.

\begin{EnvFullwidth}
\begin{solutionorgrid}[3in]
Foo.
\end{solutionorgrid}
\end{EnvFullwidth}

\part[2]
Find the distance from $E$ to $P_1$.

\begin{EnvFullwidth}
\begin{solutionorgrid}[3in]
Foo.
\end{solutionorgrid}
\end{EnvFullwidth}

\uplevel{The equations of $P_1$ and $P_2$ are used to model two hillsides that meet at a river, as shown in Figure~\ref{fig-three-planes-modelling}. The river is modelled by the line where the two planes meet. A straight bridge, modelled by $\ell$ connects $C$ to $D$.}

\begin{figure}[h]
	\centering
% 	\includegraphics[scale=.5]{0-three-planes-modelling}
	\caption{A model.}
	\label{fig-three-planes-modelling}
\end{figure}

\part[2]
The point $E$ on the bridge must be at least $1$~unit from $P_1$ and at least $1$~unit from $P_3$. Show whether or not the model satisfies the conditions.

\begin{EnvFullwidth}
\begin{solutionorgrid}[3in]
Foo.
\end{solutionorgrid}
\end{EnvFullwidth}

\end{parts}

\triast

\question % 2014 Exam, Q10.
Foo.

\begin{figure}[h]
    \centering
    \tdplotsetmaincoords{50}{195}
    \begin{tikzpicture}[scale=1,tdplot_main_coords]
    \coordinate (A) at (4,1,-2);
    \coordinate (B) at (6,3,0);
    \coordinate (C) at (8,-1,2);
    \draw[semithick,->] (0,0,0) -- (10,0,0) node[anchor=north east]{$X$};
    \draw[semithick,->] (0,0,0) -- (0,6,0) node[anchor=north west]{$Y$};
    \draw[semithick,->] (0,0,0) -- (0,0,5) node[anchor=south]{$Z$};
    \draw[semithick,dashed] (0,0,0) -- (-7,0,0);
    \draw[semithick,dashed] (0,0,0) -- (0,-7,0);
    \draw[semithick,dashed] (0,0,0) -- (0,0,-3);
    \fill (A) circle (2pt) node[anchor=north]{$A$};
    \fill (B) circle (2pt) node[anchor=north]{$B$};
    \fill (C) circle (2pt) node[anchor=south]{$C$};
    % \def\x{.5}
    % \draw[thin] (0,0,0) -- ({1.2*\x},{sqrt(3)*1.2*\x},0) node[below] {$y=\sqrt{3}x$};
    % \filldraw[
    %     draw=red,%
    %     fill=red!20,%
    % ]          (0,0,0)
    %         -- (\x,{sqrt(3)*\x},0)
    %         -- (\x,{sqrt(3)*\x},1)
    %         -- (0,0,1)
    %         -- cycle;
    \end{tikzpicture}
    \caption{Caption}
    \label{fig:my_label}
\end{figure}
